%\documentclass[letterpaper,12pt]{article}
\documentclass[10pt]{book}

%-- imports
% QRK
\newcommand{\runDate}{202007}
\newcommand{\multipleNum}{7}
\newcommand{\projectNum}{9}
\newcommand{\iterationNumber}{1}

% Set Specific Variables
\newcommand{\setNumberStart}{0}
\newcommand{\setNumberEnd}{21}
\newcommand{\cardNumberStart}{7}
\newcommand{\cardNumberEnd}{7}

% Card Size
\newcommand{\cardHeight}{4}
\newcommand{\cardWidth}{5.5}

% Cropmarks with Paper Size
\newcommand{\cropPlusCardHeight}{101.6}%4} -101.6
\newcommand{\cropPlusCardWidth}{140} %5.5 - 152.4

% page dimensions
\newcommand{\halfHeight}{2.125} %1.9 for box!
\newcommand{\halfWidth}{2.75}  %2.8

% --- Packages
\usepackage[paperwidth= \cardWidth in, paperheight=\cardHeight in,top=0in, bottom=0in, left=0in, right=0in]{geometry}

\usepackage{tikz} 			%draw

\usepackage{fontspec}		%Set font
\setmainfont{Orbit}         % special by Evan Pittson Design. If you don't have this font, use something on your computer. Or comment out this line.

\usepackage{ifthen}			% if then else

\usepackage{mathtools}

% --- Formatting
\pagenumbering{gobble}
\setlength{\parindent}{0pt}


% --- Defined Values

% Constants
\newcommand{\altMargin}{1}
\newcommand{\lrMarginStart}{2.6}
\newcommand{\udMarginStart}{2}
\newcommand{\senderNAnchor}{1.4}
\newcommand{\recipientSAnchor}{-1.25}
\newcommand{\sendInstructionsX}{-2.25}
\newcommand{\sendInstructionsY}{1.35}
\newcommand{\recipInstructionsX}{2.25}
\newcommand{\recipInstructionsY}{-1.175}
% Card 19
\newcommand{\sendInsX}{-\lrMarginStart+0.2}
\newcommand{\sendInsY}{1.35}
\newcommand{\recipOneInsX}{\lrMarginStart-0.2}
\newcommand{\recipOneInsY}{-0.25}
\newcommand{\recipTwoInsX}{-.5}
\newcommand{\recipTwoInsY}{-0.75}

\newcommand{\sender}{Sender}
\newcommand{\recipient}{Recipient}


% Styling
\newcommand{ \cardTitleSty }[1]{\raggedright \large \MakeUppercase #1}
\newcommand{ \personSty }[1]{\raggedright \normalsize \MakeUppercase #1}
\newcommand{ \altPersonSty }[1]{\raggedright \tiny \MakeUppercase #1}
% Instructions
\newcommand{\instructionFormatting}[1]{\small \MakeLowercase #1}
\newcommand{\altInstructionFormatting}[1]{\tiny \MakeLowercase #1}


\begin{document}
	\foreach \card in {\cardNumberStart,...,\cardNumberEnd}
	{
		 % --- Set sender value	and  recipient value
		\ifthenelse{\card < 1}
		{ % for preamble cards
			\newcommand{\senderOn}{0}
			\newcommand{\recipientOn}{0}
		}{ % for sender cards
			\ifthenelse{\card < 8}
			{
			\newcommand{\senderOn}{1}
			\newcommand{\recipientOn}{0}
			}{ % for recipient cards
			   \ifthenelse{\card < 14}
				{
				\newcommand{\senderOn}{0}
				\newcommand{\recipientOn}{1}
			    }{ % for joint cards
				    \newcommand{\senderOn}{1}
					\newcommand{\recipientOn}{1}
				 }
				}
			}	
		
		% - Cards with one line: 7, 8, 9
		\ifthenelse{ \card = 7 \OR \card = 8 \OR \card = 9
		}{ % Set individual card details

			% card 7  - sender 1 liner
			\ifthenelse{\card = 7}
			{
				% card 7 constants
				\newcommand{\cardTitle}{quintuplets}
\newcommand{\instructionsOne}{Make each `fifth' match}

				% generate card 7 set
				% generate card 1 for each set
\foreach \set in {\setNumberStart,...,\setNumberEnd}
{	\begin{center}
	\begin{figure}
		\resizebox{\paperwidth}{\paperheight}
		{%Generate QRK Number
\newcommand{\qrkNumber}{QRK$\;$M$\multipleNum_{\iterationNumber}\;\runDate:
  \begin{matrix*}[l]
    \card \in \{ \cardNumberStart, \dots , \cardNumberEnd\} = S_{\set}\\
    S_{\set} \subset \{ S_\setNumberStart, \dots , S_{\setNumberEnd}\}
  \end{matrix*}$}
\begin{tikzpicture}[]

    	% title
    	\node[anchor=north west] at (-\lrMarginStart,\udMarginStart)
        { \cardTitleSty{\cardTitle}};

    	% QRK Number
    	\node[anchor=south west] at (-\lrMarginStart, -\udMarginStart)
        { \tiny \qrkNumber};

    	% sender label
    	\ifthenelse{\senderOn = 1}
    					 % for the sender
    					 {\node[anchor=north east, rotate = 90, text = black!50!white] at
               (-\lrMarginStart,\senderNAnchor) {\personSty{\sender}};}
    					 % not for sender
    					 {\node[anchor=north east, rotate = 90, text = black!20!white] at
               (-\lrMarginStart,\senderNAnchor) {\personSty{\sender}};}

    	% recipient label
    	\ifthenelse{\recipientOn = 1}
    					 % for the recipient
    					 {\node[anchor=north east, rotate = -90, text = black!50!white ]
               at (\lrMarginStart, \recipientSAnchor) {\personSty{\recipient}};}
    					 % not for recipient
    					 {\node[anchor=north east, rotate = -90, text = black!20!white ]
               at (\lrMarginStart,\recipientSAnchor) {\personSty{\recipient}};}

    	% instructions
    	\node[anchor = north west] at (\sendInstructionsX, \sendInstructionsY)
        { \instructionFormatting{\instructionsOne} } ;

\end{tikzpicture}
}
	\end{figure}
\end{center}
}

			}
			% card 8 - recipient 1 liner
			{ \ifthenelse{\card =8}
				{
					% card 8 constants
					\newcommand{\cardTitle}{quintuplets}
\newcommand{\instructionsOne}{Make each `fifth' match}

					% generate card 8 set
					% generate card 1 for each set
\foreach \set in {\setNumberStart,...,\setNumberEnd}
{	\begin{center}
	\begin{figure}
		\resizebox{\cropedBoundariesW mm}{\cropedBoundariesH mm}
		{%Generate QRK Number
\newcommand{\qrkNumber}{QRK$\;$M$\multipleNum_{\iterationNumber}\;\runDate:
  \begin{matrix*}[l]
    \card \in \{ \cardNumberStart, \dots , \cardNumberEnd\} = S_{\set}\\
    S_{\set} \subset \{ S_\setNumberStart, \dots , S_{\setNumberEnd}\}
  \end{matrix*}$}
\begin{tikzpicture}[]

    	% title
    	\node[anchor=north west] at (-\lrMarginStart,\udMarginStart)
        { \cardTitleSty{\cardTitle}};

    	% QRK Number
    	\node[anchor=south west] at (-\lrMarginStart, -\udMarginStart)
        { \tiny \qrkNumber};

    	% sender label
    	\ifthenelse{\senderOn = 1}
    					 % for the sender
    					 {\node[anchor=north east, rotate = 90, text = black!50!white]
                at (-\lrMarginStart,\senderNAnchor) {\personSty{\sender}};}
    					 % not for sender
    					 {\node[anchor=north east, rotate = 90, text = black!20!white]
                at (-\lrMarginStart,\senderNAnchor) {\personSty{\sender}};}

    	% recipient label
    	\ifthenelse{\recipientOn = 1}
    					 % for the recipient
    					 {\node[anchor=north east, rotate = -90, text = black!50!white ]
                at (\lrMarginStart,\recipientSAnchor) {\personSty{\recipient}};}
    					 % not for recipient
    					 {\node[anchor=north east, rotate = -90, text = black!20!white ]
                at (\lrMarginStart,\recipientSAnchor) {\personSty{\recipient}};}

    	% instructions
    	\node[anchor = south east] at (\recipInstructionsX, \recipInstructionsY)
        { \instructionFormatting{\instructionsOne} } ;

\end{tikzpicture}
}
	\end{figure}
\end{center}
}

				}
				% card 9 - recipent 1 liner
				{	% card 8 constants
					\newcommand{\cardTitle}{ghostly tellings}
\newcommand{\instructionsOne}{to be read by a ghost}

					% generate card 9 set
					% generate card 1 for each set
\foreach \set in {\setNumberStart,...,\setNumberEnd}
{	\begin{center}
	\begin{figure}
		\resizebox{\cropedBoundariesW mm}{\cropedBoundariesH mm}
		{%Generate QRK Number
\newcommand{\qrkNumber}{QRK$\;$M$\multipleNum_{\iterationNumber}\;\runDate:
  \begin{matrix*}[l]
    \card \in \{ \cardNumberStart, \dots , \cardNumberEnd\} = S_{\set}\\
    S_{\set} \subset \{ S_\setNumberStart, \dots , S_{\setNumberEnd}\}
  \end{matrix*}$}
\begin{tikzpicture}[]

    	% title
    	\node[anchor=north west] at (-\lrMarginStart,\udMarginStart)
        { \cardTitleSty{\cardTitle}};

    	% QRK Number
    	\node[anchor=south west] at (-\lrMarginStart, -\udMarginStart)
        { \tiny \qrkNumber};

    	% sender label
    	\ifthenelse{\senderOn = 1}
    					 % for the sender
    					 {\node[anchor=north east, rotate = 90, text = black!50!white]
                at (-\lrMarginStart,\senderNAnchor) {\personSty{\sender}};}
    					 % not for sender
    					 {\node[anchor=north east, rotate = 90, text = black!20!white]
                at (-\lrMarginStart,\senderNAnchor) {\personSty{\sender}};}

    	% recipient label
    	\ifthenelse{\recipientOn = 1}
    					 % for the recipient
    					 {\node[anchor=north east, rotate = -90, text = black!50!white ]
                at (\lrMarginStart,\recipientSAnchor) {\personSty{\recipient}};}
    					 % not for recipient
    					 {\node[anchor=north east, rotate = -90, text = black!20!white ]
                at (\lrMarginStart,\recipientSAnchor) {\personSty{\recipient}};}

    	% instructions
    	\node[anchor = south east] at (\recipInstructionsX, \recipInstructionsY)
        { \instructionFormatting{\instructionsOne} } ;

\end{tikzpicture}
}
	\end{figure}
\end{center}
}

				}
			}
		} % End if one line cards
		{ % Begin Else -> != one line cards
			% - 2 line sender cards: 4, 5,  6,
			\ifthenelse{ \card = 4 \OR \card = 5 \OR \card = 6}
			{ % If card = 2 line sender card
				% card 4
				\ifthenelse{\card = 4}
				{ 	% if card = card 4 
					\newcommand{\cardTitle}{Bit By Bit}%Value Accrued
\newcommand{\instructionsOne}{each day until \MakeUppercase{I}'m complete:}
\newcommand{\instructionsTwo}{add content}

					% generate card 4 set
					% generate card 1 for each set
\foreach \set in {\setNumberStart,...,\setNumberEnd}
{	\begin{center}
	\begin{figure}
		\resizebox{\paperwidth}{\paperheight}
		{%Generate QRK Number
\newcommand{\qrkNumber}{QRK$\;$M$\multipleNum_{\iterationNumber}\;\runDate:
  \begin{matrix*}[l]
    \card \in \{ \cardNumberStart, \dots , \cardNumberEnd\} = S_{\set}\\
    S_{\set} \subset \{ S_\setNumberStart, \dots , S_{\setNumberEnd}\}
  \end{matrix*}$}
\begin{tikzpicture}[]

    	% title
    	\node[anchor=north west] at (-\lrMarginStart,\udMarginStart)
        { \cardTitleSty{\cardTitle}};

    	% QRK Number
    	\node[anchor=south west] at (-\lrMarginStart, -\udMarginStart)
        { \tiny \qrkNumber};

    	% sender label
    	\ifthenelse{\senderOn = 1}
    					 % for the sender
    					 {\node[anchor=north east, rotate = 90, text = black!50!white]
                at (-\lrMarginStart,\senderNAnchor) {\personSty{\sender}};}
    					 % not for sender
    					 {\node[anchor=north east, rotate = 90, text = black!20!white]
                at (-\lrMarginStart,\senderNAnchor) {\personSty{\sender}};}

    	% recipient label
    	\ifthenelse{\recipientOn = 1}
    					 % for the recipient
    					 {\node[anchor=north east, rotate = -90, text = black!50!white ]
               at (\lrMarginStart,\recipientSAnchor) {\personSty{\recipient}};}
    					 % not for recipient
    					 {\node[anchor=north east, rotate = -90, text = black!20!white ]
               at (\lrMarginStart,\recipientSAnchor) {\personSty{\recipient}};}

    	% instructions
    	\node[anchor = north west] at (\sendInstructionsX, \sendInstructionsY)
        { \instructionFormatting{\instructionsOne} } ;
    	\node[anchor = north west] at (\sendInstructionsX, \sendInstructionsY-0.5)
        { \instructionFormatting{\instructionsTwo} } ;

\end{tikzpicture}
}
	\end{figure}
\end{center}
}

				}	% END if card = card 4 
				{ 	% ELSE Card is a 2 line sender card != 4
					\ifthenelse{\card = 5}
					{	% if card = card 5 
						\newcommand{\cardTitle}{Behind the times}%Anachronistic Advice
\newcommand{\instructionsOne}{send advice to recipient for}
\newcommand{\instructionsTwo}{navigating a past year}

						% generate card 5 set
						% generate card 1 for each set
\foreach \set in {\setNumberStart,...,\setNumberEnd}
{	\begin{center}
	\begin{figure}
		\resizebox{\paperwidth}{\paperheight}
		{%Generate QRK Number
\newcommand{\qrkNumber}{QRK$\;$M$\multipleNum_{\iterationNumber}\;\runDate:
  \begin{matrix*}[l]
    \card \in \{ \cardNumberStart, \dots , \cardNumberEnd\} = S_{\set}\\
    S_{\set} \subset \{ S_\setNumberStart, \dots , S_{\setNumberEnd}\}
  \end{matrix*}$}
\begin{tikzpicture}[]

    	% title
    	\node[anchor=north west] at (-\lrMarginStart,\udMarginStart)
        { \cardTitleSty{\cardTitle}};

    	% QRK Number
    	\node[anchor=south west] at (-\lrMarginStart, -\udMarginStart)
        { \tiny \qrkNumber};

    	% sender label
    	\ifthenelse{\senderOn = 1}
    					 % for the sender
    					 {\node[anchor=north east, rotate = 90, text = black!50!white]
                at (-\lrMarginStart,\senderNAnchor) {\personSty{\sender}};}
    					 % not for sender
    					 {\node[anchor=north east, rotate = 90, text = black!20!white]
                at (-\lrMarginStart,\senderNAnchor) {\personSty{\sender}};}

    	% recipient label
    	\ifthenelse{\recipientOn = 1}
    					 % for the recipient
    					 {\node[anchor=north east, rotate = -90, text = black!50!white ]
               at (\lrMarginStart,\recipientSAnchor) {\personSty{\recipient}};}
    					 % not for recipient
    					 {\node[anchor=north east, rotate = -90, text = black!20!white ]
               at (\lrMarginStart,\recipientSAnchor) {\personSty{\recipient}};}

    	% instructions
    	\node[anchor = north west] at (\sendInstructionsX, \sendInstructionsY)
        { \instructionFormatting{\instructionsOne} } ;
    	\node[anchor = north west] at (\sendInstructionsX, \sendInstructionsY-0.5)
        { \instructionFormatting{\instructionsTwo} } ;

\end{tikzpicture}
}
	\end{figure}
\end{center}
}

					}	% END if card = card 5
					{	% ELSE Card is a 2 line sender card != 4 OR 5
						% card = card 6 
						\newcommand{\cardTitle}{Water Works}
\newcommand{\instructionsOne}{Use water in the filling}
\newcommand{\instructionsTwo}{of this post card}

						% generate card 6 set
						% generate card 1 for each set
\foreach \set in {\setNumberStart,...,\setNumberEnd}
{	\begin{center}
	\begin{figure}
		\resizebox{\paperwidth}{\paperheight}
		{%Generate QRK Number
\newcommand{\qrkNumber}{QRK$\;$M$\multipleNum_{\iterationNumber}\;\runDate:
  \begin{matrix*}[l]
    \card \in \{ \cardNumberStart, \dots , \cardNumberEnd\} = S_{\set}\\
    S_{\set} \subset \{ S_\setNumberStart, \dots , S_{\setNumberEnd}\}
  \end{matrix*}$}
\begin{tikzpicture}[]

    	% title
    	\node[anchor=north west] at (-\lrMarginStart,\udMarginStart)
        { \cardTitleSty{\cardTitle}};

    	% QRK Number
    	\node[anchor=south west] at (-\lrMarginStart, -\udMarginStart)
        { \tiny \qrkNumber};

    	% sender label
    	\ifthenelse{\senderOn = 1}
    					 % for the sender
    					 {\node[anchor=north east, rotate = 90, text = black!50!white]
                at (-\lrMarginStart,\senderNAnchor) {\personSty{\sender}};}
    					 % not for sender
    					 {\node[anchor=north east, rotate = 90, text = black!20!white]
                at (-\lrMarginStart,\senderNAnchor) {\personSty{\sender}};}

    	% recipient label
    	\ifthenelse{\recipientOn = 1}
    					 % for the recipient
    					 {\node[anchor=north east, rotate = -90, text = black!50!white ]
               at (\lrMarginStart,\recipientSAnchor) {\personSty{\recipient}};}
    					 % not for recipient
    					 {\node[anchor=north east, rotate = -90, text = black!20!white ]
               at (\lrMarginStart,\recipientSAnchor) {\personSty{\recipient}};}

    	% instructions
    	\node[anchor = north west] at (\sendInstructionsX, \sendInstructionsY)
        { \instructionFormatting{\instructionsOne} } ;
    	\node[anchor = north west] at (\sendInstructionsX, \sendInstructionsY-0.5)
        { \instructionFormatting{\instructionsTwo} } ;

\end{tikzpicture}
}
	\end{figure}
\end{center}
}

					}	%END ELSE Card is a 2 line sender card != 4 OR 5
				} 	% END ELSE Card is a 2 line sender card != 4
			} % End If card = 2 line sender card
			{	% BEGIN Card Has Multiple Lines 
					% - Not 2 line Sender
				\ifthenelse{\card=11 \OR \card=12}
				{	% BEGIN if card = 2 line recipient: 11, 12
					\ifthenelse{\card=11}
					{	%if card = card 11
						\newcommand{\cardTitle}{Water Works}
\newcommand{\instructionsOne}{Use water in the filling}
\newcommand{\instructionsTwo}{of this post card}

						% generate card 11 set
						% generate card 1 for each set
\foreach \set in {\setNumberStart,...,\setNumberEnd}
{	\begin{center}
	\begin{figure}
		\resizebox{\paperwidth}{\paperheight}
		{%Generate QRK Number
\newcommand{\qrkNumber}
  {QRK$\;$M$\multipleNum_{\iterationNumber}\;\runDate:
  \begin{matrix*}[l]
    \card \in  S_{\set}  | \{ \cardNumberStart, \dots , \cardNumberEnd\} = S_{\set}\\
    S_{\set} \subset \{ S_\setNumberStart, \dots , S_{\setNumberEnd}\}
  \end{matrix*}$}
\begin{tikzpicture}[]

    	% title
    	\node[anchor=north west] at (-\lrMarginStart,\udMarginStart)
        {\cardTitleSty{\cardTitle}};

    	% QRK Number
    	\node[anchor=south west] at (-\lrMarginStart, -\udMarginStart)
      {\tiny \qrkNumber};

    	% sender label
    	\ifthenelse{\senderOn = 1}
    					 % for the sender
    					 {\node[anchor=north east, rotate = 90, text = black!50!white] at
               (-\lrMarginStart,\senderNAnchor) {\personSty{\sender}};}
    					 % not for sender
    					 {\node[anchor=north east, rotate = 90, text = black!20!white] at
               (-\lrMarginStart,\senderNAnchor) {\personSty{\sender}};}

    	% recipient label
    	\ifthenelse{\recipientOn = 1}
    					 % for the recipient
    					 {\node[anchor=north east, rotate = -90, text = black!50!white ]
               at (\lrMarginStart, \recipientSAnchor) {\personSty{\recipient}};}
    					 % not for recipient
    					 {\node[anchor=north east, rotate = -90, text = black!20!white ]
               at (\lrMarginStart,\recipientSAnchor) {\personSty{\recipient}};}

    	% instructions
    	\node[anchor= south east] at (\recipInstructionsX,\recipInstructionsY+0.5)
        {\instructionFormatting{\instructionsOne}} ;
			\node[anchor= south east] at (\recipInstructionsX,\recipInstructionsY)
        {\instructionFormatting{\instructionsTwo}} ;

\end{tikzpicture}
}
	\end{figure}
\end{center}
}

					}	% END if card = card 11
					{	% card = card 12
						\newcommand{\cardTitle}{Love all}
\newcommand{\instructionsOne}{Share me with the}
\newcommand{\instructionsTwo}{next person you meet}

						% generate card 12 set
						% generate card 1 for each set
\foreach \set in {\setNumberStart,...,\setNumberEnd}
{	\begin{center}
	\begin{figure}
		\resizebox{\paperwidth}{\paperheight}
		{%Generate QRK Number
\newcommand{\qrkNumber}
  {QRK$\;$M$\multipleNum_{\iterationNumber}\;\runDate:
  \begin{matrix*}[l]
    \card \in  S_{\set}  | \{ \cardNumberStart, \dots , \cardNumberEnd\} = S_{\set}\\
    S_{\set} \subset \{ S_\setNumberStart, \dots , S_{\setNumberEnd}\}
  \end{matrix*}$}
\begin{tikzpicture}[]

    	% title
    	\node[anchor=north west] at (-\lrMarginStart,\udMarginStart)
        {\cardTitleSty{\cardTitle}};

    	% QRK Number
    	\node[anchor=south west] at (-\lrMarginStart, -\udMarginStart)
      {\tiny \qrkNumber};

    	% sender label
    	\ifthenelse{\senderOn = 1}
    					 % for the sender
    					 {\node[anchor=north east, rotate = 90, text = black!50!white] at
               (-\lrMarginStart,\senderNAnchor) {\personSty{\sender}};}
    					 % not for sender
    					 {\node[anchor=north east, rotate = 90, text = black!20!white] at
               (-\lrMarginStart,\senderNAnchor) {\personSty{\sender}};}

    	% recipient label
    	\ifthenelse{\recipientOn = 1}
    					 % for the recipient
    					 {\node[anchor=north east, rotate = -90, text = black!50!white ]
               at (\lrMarginStart, \recipientSAnchor) {\personSty{\recipient}};}
    					 % not for recipient
    					 {\node[anchor=north east, rotate = -90, text = black!20!white ]
               at (\lrMarginStart,\recipientSAnchor) {\personSty{\recipient}};}

    	% instructions
    	\node[anchor= south east] at (\recipInstructionsX,\recipInstructionsY+0.5)
        {\instructionFormatting{\instructionsOne}} ;
			\node[anchor= south east] at (\recipInstructionsX,\recipInstructionsY)
        {\instructionFormatting{\instructionsTwo}} ;

\end{tikzpicture}
}
	\end{figure}
\end{center}
}

					}	% END card = card 12
				}	% END if card = 2 line recipient: 11, 12
				% ELSE if card != 2 line recipient
				{	% Begin 3 Line or More cards
					% - 3 line sender: 1, 3
					\ifthenelse{\card=1 \OR \card=3}
					{	% Begin if Card = 3 line sender: 1, 3
						\ifthenelse{\card=1}
						{	% if card = card 1
							\newcommand{\cardTitle}{olfactory unity}
\newcommand{\tripleOne}{lay me in a herb}
\newcommand{\tripleTwo}{smell me}
\newcommand{\tripleThree}{fill me}
							% generate card 1
							% generate card 1 for each set
\foreach \set in {\setNumberStart,...,\setNumberEnd}
{	\begin{center}
		\begin{figure}
			\resizebox{\paperwidth}{\paperheight}
			{%Generate QRK Number
\newcommand{\qrkNumber}
  {QRK$\;$M$\multipleNum_{\iterationNumber}\;\runDate:
  \begin{matrix*}[l]
    \card \in  S_{\set}  | \{ \cardNumberStart, \dots , \cardNumberEnd\} = S_{\set}\\
    S_{\set} \subset \{ S_\setNumberStart, \dots , S_{\setNumberEnd}\}
  \end{matrix*}$}
\begin{tikzpicture}[]

    	% title
    	\node[anchor=north west] at (-\lrMarginStart,\udMarginStart)
        { \cardTitleSty{\cardTitle}};

    	% QRK Number
    	\node[anchor=south west] at (-\lrMarginStart, -\udMarginStart)
        { \tiny \qrkNumber};

    	% sender label
    	\ifthenelse{\senderOn = 1}
    					 % for the sender
    					 {\node[anchor=north east, rotate = 90, text = black!50!white]
                at (-\lrMarginStart,\senderNAnchor) {\personSty{\sender}};}
    					 % not for sender
    					 {\node[anchor=north east, rotate = 90, text = black!20!white]
                at (-\lrMarginStart,\senderNAnchor) {\personSty{\sender}};}

    	% recipient label
    	\ifthenelse{\recipientOn = 1}
    					 % for the recipient
    					 {\node[anchor=north east, rotate = -90, text = black!50!white ]
               at (\lrMarginStart, \recipientSAnchor) {\personSty{\recipient}};}
    					 % not for recipient
    					 {\node[anchor=north east, rotate = -90, text = black!20!white ]
               at (\lrMarginStart,\recipientSAnchor) {\personSty{\recipient}};}

    	% instructions
    	\node[anchor = north west] at (\sendInstructionsX, \sendInstructionsY)
        {\instructionFormatting{\instructionsOne} } ;
    	\node[anchor = north west] at (\sendInstructionsX, \sendInstructionsY-0.5)
        {\instructionFormatting{\instructionsTwo}} ;
    	\node[anchor = north west] at (\sendInstructionsX, \sendInstructionsY-1)
        {\instructionFormatting{\instructionsThree}} ;

\end{tikzpicture}
}
		\end{figure}
	\end{center}
}

						}	% END if card = card 11
						{	% if card = card 3
							\newcommand{\cardTitle}{rainbow}
\newcommand{\instructionsOne}{gather colors}
\newcommand{\instructionsTwo}{choose 1 color and adorn me}
\newcommand{\instructionsThree}{repeat}

							% generate card 3
							% generate card 1 for each set
\foreach \set in {\setNumberStart,...,\setNumberEnd}
{	\begin{center}
		\begin{figure}
			\resizebox{\paperwidth}{\paperheight}
			{%Generate QRK Number
\newcommand{\qrkNumber}
  {QRK$\;$M$\multipleNum_{\iterationNumber}\;\runDate:
  \begin{matrix*}[l]
    \card \in  S_{\set}  | \{ \cardNumberStart, \dots , \cardNumberEnd\} = S_{\set}\\
    S_{\set} \subset \{ S_\setNumberStart, \dots , S_{\setNumberEnd}\}
  \end{matrix*}$}
\begin{tikzpicture}[]

    	% title
    	\node[anchor=north west] at (-\lrMarginStart,\udMarginStart)
        { \cardTitleSty{\cardTitle}};

    	% QRK Number
    	\node[anchor=south west] at (-\lrMarginStart, -\udMarginStart)
        { \tiny \qrkNumber};

    	% sender label
    	\ifthenelse{\senderOn = 1}
    					 % for the sender
    					 {\node[anchor=north east, rotate = 90, text = black!50!white]
                at (-\lrMarginStart,\senderNAnchor) {\personSty{\sender}};}
    					 % not for sender
    					 {\node[anchor=north east, rotate = 90, text = black!20!white]
                at (-\lrMarginStart,\senderNAnchor) {\personSty{\sender}};}

    	% recipient label
    	\ifthenelse{\recipientOn = 1}
    					 % for the recipient
    					 {\node[anchor=north east, rotate = -90, text = black!50!white ]
               at (\lrMarginStart, \recipientSAnchor) {\personSty{\recipient}};}
    					 % not for recipient
    					 {\node[anchor=north east, rotate = -90, text = black!20!white ]
               at (\lrMarginStart,\recipientSAnchor) {\personSty{\recipient}};}

    	% instructions
    	\node[anchor = north west] at (\sendInstructionsX, \sendInstructionsY)
        {\instructionFormatting{\instructionsOne} } ;
    	\node[anchor = north west] at (\sendInstructionsX, \sendInstructionsY-0.5)
        {\instructionFormatting{\instructionsTwo}} ;
    	\node[anchor = north west] at (\sendInstructionsX, \sendInstructionsY-1)
        {\instructionFormatting{\instructionsThree}} ;

\end{tikzpicture}
}
		\end{figure}
	\end{center}
}

						}	% END if card = card 3
					}	% END if Card = 3 line sender: 1, 3
					{	% Card has 3 or more lines and is not sender
						\ifthenelse{\card=10 \OR \card=13}
						{	% BEGIN IF Card = 3 line recipient:10, 13
							\ifthenelse{\card = 10}
							{	% if = card 10
								\newcommand{\cardTitle}{Love all}
\newcommand{\instructionsOne}{Share me with the}
\newcommand{\instructionsTwo}{next person you meet}

								% generate 3 line recipient cards
								% generate card 1 for each set
\foreach \set in {\setNumberStart,...,\setNumberEnd}
{	\begin{center}
	\begin{figure}
		\resizebox{\paperwidth}{\paperheight}
		{%Generate QRK Number
\newcommand{\qrkNumber}
  {QRK$\;$M$\multipleNum_{\iterationNumber}\;\runDate:
  \begin{matrix*}[l]
    \card \in  S_{\set}  | \{ \cardNumberStart, \dots , \cardNumberEnd\} = S_{\set}\\
    S_{\set} \subset \{ S_\setNumberStart, \dots , S_{\setNumberEnd}\}
  \end{matrix*}$}
\begin{tikzpicture}[]

    	% title
    	\node[anchor=north west] at (-\lrMarginStart,\udMarginStart)
        { \cardTitleSty{\cardTitle}};

    	% QRK Number
    	\node[anchor=south west] at (-\lrMarginStart, -\udMarginStart)
        { \tiny \qrkNumber};

    	% sender label
    	\ifthenelse{\senderOn = 1}
    					 % for the sender
    					 {\node[anchor=north east, rotate = 90, text = black!50!white]
                at (-\lrMarginStart,\senderNAnchor) {\personSty{\sender}};}
    					 % not for sender
    					 {\node[anchor=north east, rotate = 90, text = black!20!white]
                at (-\lrMarginStart,\senderNAnchor) {\personSty{\sender}};}

    	% recipient label
    	\ifthenelse{\recipientOn = 1}
    					 % for the recipient
    					 {\node[anchor=north east, rotate = -90, text = black!50!white ]
              at (\lrMarginStart, \recipientSAnchor) {\personSty{\recipient}};}
    					 % not for recipient
    					 {\node[anchor=north east, rotate = -90, text = black!20!white ]
               at (\lrMarginStart,\recipientSAnchor) {\personSty{\recipient}};}

    	% instructions
    	\node[anchor = south east] at (\recipInstructionsX, \recipInstructionsY+1)
       	 	{ \instructionFormatting{\instructionsOne} } ;
    	\node[anchor = south east] at (\recipInstructionsX, \recipInstructionsY+0.5)
        	{ \instructionFormatting{\instructionsTwo} } ;
		\node[anchor = south east] at (\recipInstructionsX, \recipInstructionsY)
        	{ \instructionFormatting{\instructionsThree} } ;

\end{tikzpicture}
}
	\end{figure}
\end{center}
}

							}	% END if = card 10
							{	% Else:  13
								\newcommand{\cardTitle}{Postcard perfect}
\newcommand{\instructionsOne}{Read me}
\newcommand{\instructionsTwo}{Adorn me}
\newcommand{\instructionsThree}{Honor me}


								% generate 3 line recipient cards
								% generate card 1 for each set
\foreach \set in {\setNumberStart,...,\setNumberEnd}
{	\begin{center}
	\begin{figure}
		\resizebox{\paperwidth}{\paperheight}
		{%Generate QRK Number
\newcommand{\qrkNumber}
  {QRK$\;$M$\multipleNum_{\iterationNumber}\;\runDate:
  \begin{matrix*}[l]
    \card \in  S_{\set}  | \{ \cardNumberStart, \dots , \cardNumberEnd\} = S_{\set}\\
    S_{\set} \subset \{ S_\setNumberStart, \dots , S_{\setNumberEnd}\}
  \end{matrix*}$}
\begin{tikzpicture}[]

    	% title
    	\node[anchor=north west] at (-\lrMarginStart,\udMarginStart)
        { \cardTitleSty{\cardTitle}};

    	% QRK Number
    	\node[anchor=south west] at (-\lrMarginStart, -\udMarginStart)
        { \tiny \qrkNumber};

    	% sender label
    	\ifthenelse{\senderOn = 1}
    					 % for the sender
    					 {\node[anchor=north east, rotate = 90, text = black!50!white]
                at (-\lrMarginStart,\senderNAnchor) {\personSty{\sender}};}
    					 % not for sender
    					 {\node[anchor=north east, rotate = 90, text = black!20!white]
                at (-\lrMarginStart,\senderNAnchor) {\personSty{\sender}};}

    	% recipient label
    	\ifthenelse{\recipientOn = 1}
    					 % for the recipient
    					 {\node[anchor=north east, rotate = -90, text = black!50!white ]
              at (\lrMarginStart, \recipientSAnchor) {\personSty{\recipient}};}
    					 % not for recipient
    					 {\node[anchor=north east, rotate = -90, text = black!20!white ]
               at (\lrMarginStart,\recipientSAnchor) {\personSty{\recipient}};}

    	% instructions
    	\node[anchor = south east] at (\recipInstructionsX, \recipInstructionsY+1)
       	 	{ \instructionFormatting{\instructionsOne} } ;
    	\node[anchor = south east] at (\recipInstructionsX, \recipInstructionsY+0.5)
        	{ \instructionFormatting{\instructionsTwo} } ;
		\node[anchor = south east] at (\recipInstructionsX, \recipInstructionsY)
        	{ \instructionFormatting{\instructionsThree} } ;

\end{tikzpicture}
}
	\end{figure}
\end{center}
}

							}	% END Else: 13
						}	% END IF Card = 3 line recipient:10, 13
						{ 	% Card has 4 lines or is a multiperson card
							\ifthenelse{\card=2}
							{	% BEGIN IF Card = 4 line sender: 2
								\newcommand{\cardTitle}{Meditative musings}olfactory unity
\newcommand{\instructionsOne}{sit comfortably}
\newcommand{\instructionsTwo}{close your eyes}
\newcommand{\instructionsThree}{breathe many times}
\newcommand{\instructionsFour}{fill me}

								% generate 4 line recipient cards
								% generate card 1 for each set
\foreach \set in {\setNumberStart,...,\setNumberEnd}
{	\begin{center}
	\begin{figure}
		\resizebox{\cropedBoundariesW mm}{\cropedBoundariesH mm}
		{%Generate QRK Number
\newcommand{\qrkNumber}
  {QRK$\;$M$\multipleNum_{\iterationNumber}\;\runDate:
  \begin{matrix*}[l]
    \card \in  S_{\set}  | \{ \cardNumberStart, \dots , \cardNumberEnd\} = S_{\set}\\
    S_{\set} \subset \{ S_\setNumberStart, \dots , S_{\setNumberEnd}\}
  \end{matrix*}$}
\begin{tikzpicture}[]

    	% title
    	\node[anchor=north west] at (-\lrMarginStart,\udMarginStart)
        { \cardTitleSty{\cardTitle}};

    	% QRK Number
    	\node[anchor=south west] at (-\lrMarginStart, -\udMarginStart)
        { \tiny \qrkNumber};

    	% sender label
    	\ifthenelse{\senderOn = 1}
    					 % for the sender
    					 {\node[anchor=north east, rotate = 90, text = black!50!white]
                at (-\lrMarginStart,\senderNAnchor) {\personSty{\sender}};}
    					 % not for sender
    					 {\node[anchor=north east, rotate = 90, text = black!20!white]
                at (-\lrMarginStart,\senderNAnchor) {\personSty{\sender}};}

    	% recipient label
    	\ifthenelse{\recipientOn = 1}
    					 % for the recipient
    					 {\node[anchor=north east, rotate = -90, text = black!50!white ]
               at (\lrMarginStart, \recipientSAnchor) {\personSty{\recipient}};}
    					 % not for recipient
    					 {\node[anchor=north east, rotate = -90, text = black!20!white ]
                at (\lrMarginStart,\recipientSAnchor) {\personSty{\recipient}};}

    	% instructions
    	\node[anchor = north west] at (\sendInstructionsX, \sendInstructionsY)
      { \instructionFormatting{\instructionsOne} } ;
    	\node[anchor = north west] at (\sendInstructionsX, \sendInstructionsY-0.5)
      { \instructionFormatting{\instructionsTwo} } ;
			\node[anchor = north west] at (\sendInstructionsX, \sendInstructionsY-1)
      { \instructionFormatting{\instructionsThree} } ;
			\node[anchor = north west] at (\sendInstructionsX, \sendInstructionsY-1.5)
      { \instructionFormatting{\instructionsFour} } ;

\end{tikzpicture}
}
	\end{figure}
\end{center}
}

							}	% END IF Card = 4 line sender: 2
							{ 	% BEGIN Multiperson Cards
								\ifthenelse{\card=17}
								{	% BEGIN if card is multiperson, 1:1 -> 17
									\newcommand{\cardTitle}{Meditative musings}olfactory unity
\newcommand{\instructionsOne}{sit comfortably}
\newcommand{\instructionsTwo}{close your eyes}
\newcommand{\instructionsThree}{breathe many times}
\newcommand{\instructionsFour}{fill me}

									% generate 1:1
									% generate card 1 for each set
\foreach \set in {\setNumberStart,...,\setNumberEnd}
{	\begin{center}
	\begin{figure}
		\resizebox{\cropedBoundariesW mm}{\cropedBoundariesH mm}
		{%Generate QRK Number
\newcommand{\qrkNumber}{QRK$\;$M$\multipleNum_{\iterationNumber}\;\runDate:\begin{matrix*}[l]
\card \in \{ \cardNumberStart, \dots , \cardNumberEnd\} = S_{\set}\\
S_{\set} \subset \{ S_\setNumberStart, \dots , S_{\setNumberEnd}\}
\end{matrix*}$}
\begin{tikzpicture}[]
		% card
    	%\draw[line width = 0.05cm] (-\halfWidth,-\halfHeight) rectangle (\halfWidth,\halfHeight);

    	% title
    	\node[anchor=north west] at (-\lrMarginStart,\udMarginStart) { \cardTitleSty{\cardTitle}};

    	% QRK Number
    	\node[anchor=south west] at (-\lrMarginStart, -\udMarginStart) { \tiny \qrkNumber};

    	% sender label
    	\ifthenelse{\senderOn = 1}
    					 % for the sender
    					 {\node[anchor=north east, rotate = 90, text = black!50!white] at (-\lrMarginStart,\senderNAnchor) {\personSty{\sender}};}
    					 % not for sender
    					 {\node[anchor=north east, rotate = 90, text = black!20!white] at (-\lrMarginStart,\senderNAnchor) {\personSty{\sender}};}

    	% recipient label
    	\ifthenelse{\recipientOn = 1}
    					 % for the recipient
    					 {\node[anchor=north east, rotate = -90, text = black!50!white ] at (\lrMarginStart, \recipientSAnchor) {\personSty{\recipient}};}
    					 % not for recipient
    					 {\node[anchor=north east, rotate = -90, text = black!20!white ] at (\lrMarginStart,\recipientSAnchor) {\personSty{\recipient}};}

    	% instructions - sender
    	\node[anchor = north west] at (\sendInstructionsX, \sendInstructionsY) { \instructionFormatting{\instructionsOneS} } ;
    	% instructions - recipient
	\node[anchor = south east] at (\recipInstructionsX, \recipInstructionsY) { \instructionFormatting{\instructionsOneR} } ;

\end{tikzpicture}
}
	\end{figure}
\end{center}
}

								}	% END if card is multiperson, 1:1 -> 17
								{	% Multiperson, not 1:1
									\ifthenelse{\card = 20}
									{	% BEGIN if card is 2:1 -> 20
										\newcommand{\cardTitle}{A mile in my shoes}
\newcommand{\instructionsOneS}{Take a walk}
\newcommand{\instructionsTwoS}{Describe your walk}
\newcommand{\instructionsOneR}{Reenact the sender's walk}

										% generate 2:1
										% generate card 1 for each set
\foreach \set in {\setNumberStart,...,\setNumberEnd}
{	\begin{center}
	\begin{figure}
		\resizebox{\paperwidth}{\paperheight}
		{%Generate QRK Number
\newcommand{\qrkNumber}
  {QRK$\;$M$\multipleNum_{\iterationNumber}\;\runDate:
  \begin{matrix*}[l]
    \card \in \{ \cardNumberStart, \dots , \cardNumberEnd\} = S_{\set}\\
    S_{\set} \subset \{ S_\setNumberStart, \dots , S_{\setNumberEnd}\}
  \end{matrix*}$}
\begin{tikzpicture}[]
    	% title
    	\node[anchor=north west] at (-\lrMarginStart,\udMarginStart)
        { \cardTitleSty{\cardTitle}};

    	% QRK Number
    	\node[anchor=south west] at (-\lrMarginStart, -\udMarginStart)
        { \tiny \qrkNumber};

    	% sender label
    	\ifthenelse{\senderOn = 1}
    					 % for the sender
    					 {\node[anchor=north east, rotate = 90, text = black!50!white]
               at (-\lrMarginStart,\senderNAnchor) {\personSty{\sender}};}
    					 % not for sender
    					 {\node[anchor=north east, rotate = 90, text = black!20!white]
               at (-\lrMarginStart,\senderNAnchor) {\personSty{\sender}};}

    	% recipient label
    	\ifthenelse{\recipientOn = 1}
    					 % for the recipient
    					 {\node[anchor=north east, rotate = -90, text = black!50!white ]
               at (\lrMarginStart, \recipientSAnchor) {\personSty{\recipient}};}
    					 % not for recipient
    					 {\node[anchor=north east, rotate = -90, text = black!20!white ]
               at (\lrMarginStart,\recipientSAnchor) {\personSty{\recipient}};}

    	% instructions - sender
    	\node[anchor = north west] at (\sendInstructionsX, \sendInstructionsY)
        { \instructionFormatting{\instructionsOneS} } ;
			\node[anchor = north west] at (\sendInstructionsX, \sendInstructionsY-0.5)
        { \instructionFormatting{\instructionsTwoS} } ;
			% instructions - recipient
	\node[anchor = south east] at (\recipInstructionsX, \recipInstructionsY)
    { \instructionFormatting{\instructionsOneR} } ;
    
\end{tikzpicture}
}
	\end{figure}
\end{center}
}

									}	% END if card is 2:1 -> 20
									{	% Multiperson, not 1:1/2:1
										\ifthenelse{\card = 16 \OR \card = 18}
										{	% Begin if Card = 2:2 -> {16, 18}
											\ifthenelse{\card=16}
											{	% if card = 16
												\newcommand{\cardTitle}{light of the moon}
\newcommand{\instructionsOneS}{Wait until night of full moon}
\newcommand{\instructionsTwoS}{Fill me by moonlight}
\newcommand{\instructionsOneR}{Wait until night of full moon}
\newcommand{\instructionsTwoR}{Enjoy me by moonlight}

												% generate 2:2
												% generate card 1 for each set
\foreach \set in {\setNumberStart,...,\setNumberEnd}
{	\begin{center}
	\begin{figure}
		\resizebox{\paperwidth}{\paperheight}
		{%Generate QRK Number
\newcommand{\qrkNumber}
  {QRK$\;$M$\multipleNum_{\iterationNumber}\;\runDate:
  \begin{matrix*}[l]
    \card \in \{ \cardNumberStart, \dots , \cardNumberEnd\} = S_{\set}\\
    S_{\set} \subset \{ S_\setNumberStart, \dots , S_{\setNumberEnd}\}
  \end{matrix*}$}
\begin{tikzpicture}[]

    	% title
    	\node[anchor=north west] at (-\lrMarginStart,\udMarginStart)
        { \cardTitleSty{\cardTitle}};

    	% QRK Number
    	\node[anchor=south west] at (-\lrMarginStart, -\udMarginStart)
        { \tiny \qrkNumber};

    	% sender label
    	\ifthenelse{\senderOn = 1}
    					 % for the sender
    					 {\node[anchor=north east, rotate = 90, text = black!50!white]
               at (-\lrMarginStart,\senderNAnchor) {\personSty{\sender}};}
    					 % not for sender
    					 {\node[anchor=north east, rotate = 90, text = black!20!white]
               at (-\lrMarginStart,\senderNAnchor) {\personSty{\sender}};}

    	% recipient label
    	\ifthenelse{\recipientOn = 1}
    					 % for the recipient
    					 {\node[anchor=north east, rotate = -90, text = black!50!white ]
               at (\lrMarginStart, \recipientSAnchor) {\personSty{\recipient}};}
    					 % not for recipient
    					 {\node[anchor=north east, rotate = -90, text = black!20!white ]
               at (\lrMarginStart,\recipientSAnchor) {\personSty{\recipient}};}

    	% instructions - sender
    	\node[anchor = north west] at (\sendInstructionsX, \sendInstructionsY)
        { \instructionFormatting{\instructionsOneS} } ;
			\node[anchor = north west] at (\sendInstructionsX, \sendInstructionsY-0.5)
        { \instructionFormatting{\instructionsTwoS} } ;
			% instructions - recipient
	\node[anchor = south east] at (\recipInstructionsX, \recipInstructionsY+0.5)
    { \instructionFormatting{\instructionsOneR} } ;
	\node[anchor = south east] at (\recipInstructionsX, \recipInstructionsY)
    { \instructionFormatting{\instructionsTwoR} } ;

\end{tikzpicture}
}
	\end{figure}
\end{center}
}

											}	
											{	% if card = 18
												\newcommand{\cardTitle}{assisted embrace}
\newcommand{\instructionsOneS}{Hug someone}
\newcommand{\instructionsTwoS}{Fill me}
\newcommand{\instructionsOneR}{Hug someone}
\newcommand{\instructionsTwoR}{Enjoy me}

												% generate 2:2
												% generate card 1 for each set
\foreach \set in {\setNumberStart,...,\setNumberEnd}
{	\begin{center}
	\begin{figure}
		\resizebox{\paperwidth}{\paperheight}
		{%Generate QRK Number
\newcommand{\qrkNumber}
  {QRK$\;$M$\multipleNum_{\iterationNumber}\;\runDate:
  \begin{matrix*}[l]
    \card \in \{ \cardNumberStart, \dots , \cardNumberEnd\} = S_{\set}\\
    S_{\set} \subset \{ S_\setNumberStart, \dots , S_{\setNumberEnd}\}
  \end{matrix*}$}
\begin{tikzpicture}[]

    	% title
    	\node[anchor=north west] at (-\lrMarginStart,\udMarginStart)
        { \cardTitleSty{\cardTitle}};

    	% QRK Number
    	\node[anchor=south west] at (-\lrMarginStart, -\udMarginStart)
        { \tiny \qrkNumber};

    	% sender label
    	\ifthenelse{\senderOn = 1}
    					 % for the sender
    					 {\node[anchor=north east, rotate = 90, text = black!50!white]
               at (-\lrMarginStart,\senderNAnchor) {\personSty{\sender}};}
    					 % not for sender
    					 {\node[anchor=north east, rotate = 90, text = black!20!white]
               at (-\lrMarginStart,\senderNAnchor) {\personSty{\sender}};}

    	% recipient label
    	\ifthenelse{\recipientOn = 1}
    					 % for the recipient
    					 {\node[anchor=north east, rotate = -90, text = black!50!white ]
               at (\lrMarginStart, \recipientSAnchor) {\personSty{\recipient}};}
    					 % not for recipient
    					 {\node[anchor=north east, rotate = -90, text = black!20!white ]
               at (\lrMarginStart,\recipientSAnchor) {\personSty{\recipient}};}

    	% instructions - sender
    	\node[anchor = north west] at (\sendInstructionsX, \sendInstructionsY)
        { \instructionFormatting{\instructionsOneS} } ;
			\node[anchor = north west] at (\sendInstructionsX, \sendInstructionsY-0.5)
        { \instructionFormatting{\instructionsTwoS} } ;
			% instructions - recipient
	\node[anchor = south east] at (\recipInstructionsX, \recipInstructionsY+0.5)
    { \instructionFormatting{\instructionsOneR} } ;
	\node[anchor = south east] at (\recipInstructionsX, \recipInstructionsY)
    { \instructionFormatting{\instructionsTwoR} } ;

\end{tikzpicture}
}
	\end{figure}
\end{center}
}

											}	% END if card = 18
										}	% End if Card = 2:2 -> {16, 18}
										{	% BEGIN Multiperson, !=  1:1/2:1/2:2
											\ifthenelse{\card=14 \OR \card=15}
											{	% Begin if card = 3:2 ->{14, 15}
												\ifthenelse{\card=14}
												{	% if card = 14
													\newcommand{\cardTitle}{a drink shared}
\newcommand{\instructionsOneS}{choose a warm drink}
\newcommand{\instructionsTwoS}{drink}
\newcommand{\instructionsThreeS}{fill me}
\newcommand{\instructionsOneR}{make a warm drink}
\newcommand{\instructionsTwoR}{examine me and drink}

													% generate 3:2
													% generate card 1 for each set
\foreach \set in {\setNumberStart,...,\setNumberEnd}
{	\begin{center}
	\begin{figure}
		\resizebox{\cropedBoundariesW mm}{\cropedBoundariesH mm}
		{%Generate QRK Number
\newcommand{\qrkNumber}
  {QRK$\;$M$\multipleNum_{\iterationNumber}\;\runDate:
  \begin{matrix*}[l]
    \card \in  S_{\set}  | \{ \cardNumberStart, \dots , \cardNumberEnd\} = S_{\set}\\
    S_{\set} \subset \{ S_\setNumberStart, \dots , S_{\setNumberEnd}\}
  \end{matrix*}$}
\begin{tikzpicture}[]

    	% title
    	\node[anchor=north west] at (-\lrMarginStart,\udMarginStart)
        { \cardTitleSty{\cardTitle}};

    	% QRK Number
    	\node[anchor=south west] at (-\lrMarginStart, -\udMarginStart)
        { \tiny \qrkNumber};

    	% sender label
    	\ifthenelse{\senderOn = 1}
    					 % for the sender
    					 {\node[anchor=north east, rotate = 90, text = black!50!white]
                at (-\lrMarginStart,\senderNAnchor) {\personSty{\sender}};}
    					 % not for sender
    					 {\node[anchor=north east, rotate = 90, text = black!20!white]
                at (-\lrMarginStart,\senderNAnchor) {\personSty{\sender}};}

    	% recipient label
    	\ifthenelse{\recipientOn = 1}
    					 % for the recipient
    					 {\node[anchor=north east, rotate = -90, text = black!50!white ]
                at (\lrMarginStart,\recipientSAnchor) {\personSty{\recipient}};}
    					 % not for recipient
    					 {\node[anchor=north east, rotate = -90, text = black!20!white ]
               at (\lrMarginStart,\recipientSAnchor) {\personSty{\recipient}};}

    	% instructions - sender
    	\node[anchor= north west] at (\sendInstructionsX,\sendInstructionsY)
        { \instructionFormatting{\instructionsOneS} } ;
			\node[anchor= north west] at (\sendInstructionsX,\sendInstructionsY-0.4)
        { \instructionFormatting{\instructionsTwoS} } ;
			\node[anchor= north west] at (\sendInstructionsX,\sendInstructionsY-0.8)
        { \instructionFormatting{\instructionsThreeS} } ;
			% instructions - recipient
	    \node[anchor= south east] at (\recipInstructionsX,\recipInstructionsY+0.4)
        { \instructionFormatting{\instructionsOneR} } ;
	    \node[anchor= south east] at (\recipInstructionsX,\recipInstructionsY)
        { \instructionFormatting{\instructionsTwoR} } ;

\end{tikzpicture}
}
	\end{figure}
\end{center}
}

												}	% END if card = 14
												{	% if card = 15
													\newcommand{\cardTitle}{Dance Duet}
\newcommand{\instructionsOneS}{Choose a song: \rule{2cm}{0.2mm}}
\newcommand{\instructionsTwoS}{listen \& dance}
\newcommand{\instructionsThreeS}{write me}
\newcommand{\instructionsOneR}{play song \& dance}
\newcommand{\instructionsTwoR}{read me}

													% generate 3:2
													% generate card 1 for each set
\foreach \set in {\setNumberStart,...,\setNumberEnd}
{	\begin{center}
	\begin{figure}
		\resizebox{\cropedBoundariesW mm}{\cropedBoundariesH mm}
		{%Generate QRK Number
\newcommand{\qrkNumber}
  {QRK$\;$M$\multipleNum_{\iterationNumber}\;\runDate:
  \begin{matrix*}[l]
    \card \in  S_{\set}  | \{ \cardNumberStart, \dots , \cardNumberEnd\} = S_{\set}\\
    S_{\set} \subset \{ S_\setNumberStart, \dots , S_{\setNumberEnd}\}
  \end{matrix*}$}
\begin{tikzpicture}[]

    	% title
    	\node[anchor=north west] at (-\lrMarginStart,\udMarginStart)
        { \cardTitleSty{\cardTitle}};

    	% QRK Number
    	\node[anchor=south west] at (-\lrMarginStart, -\udMarginStart)
        { \tiny \qrkNumber};

    	% sender label
    	\ifthenelse{\senderOn = 1}
    					 % for the sender
    					 {\node[anchor=north east, rotate = 90, text = black!50!white]
                at (-\lrMarginStart,\senderNAnchor) {\personSty{\sender}};}
    					 % not for sender
    					 {\node[anchor=north east, rotate = 90, text = black!20!white]
                at (-\lrMarginStart,\senderNAnchor) {\personSty{\sender}};}

    	% recipient label
    	\ifthenelse{\recipientOn = 1}
    					 % for the recipient
    					 {\node[anchor=north east, rotate = -90, text = black!50!white ]
                at (\lrMarginStart,\recipientSAnchor) {\personSty{\recipient}};}
    					 % not for recipient
    					 {\node[anchor=north east, rotate = -90, text = black!20!white ]
               at (\lrMarginStart,\recipientSAnchor) {\personSty{\recipient}};}

    	% instructions - sender
    	\node[anchor= north west] at (\sendInstructionsX,\sendInstructionsY)
        { \instructionFormatting{\instructionsOneS} } ;
			\node[anchor= north west] at (\sendInstructionsX,\sendInstructionsY-0.4)
        { \instructionFormatting{\instructionsTwoS} } ;
			\node[anchor= north west] at (\sendInstructionsX,\sendInstructionsY-0.8)
        { \instructionFormatting{\instructionsThreeS} } ;
			% instructions - recipient
	    \node[anchor= south east] at (\recipInstructionsX,\recipInstructionsY+0.4)
        { \instructionFormatting{\instructionsOneR} } ;
	    \node[anchor= south east] at (\recipInstructionsX,\recipInstructionsY)
        { \instructionFormatting{\instructionsTwoR} } ;

\end{tikzpicture}
}
	\end{figure}
\end{center}
}

												}	% END if card = 15
											}	% END if card = 3:2 ->{14, 15}
											{	% BEGIN Multiperson, !=  1:1/2:1/2:2/3:2
												%	 3:2:1 -> card 19
												\newcommand{\cardTitle}{collective memory}
\newcommand{\instructionsOneS}{list 2 people who share a memory}
\newcommand{\instructionsTwoS}{write about shared memory}
\newcommand{\instructionsThreeS}{send to recipient 1}
\newcommand{\instructionsOneROne}{write about same memory}
\newcommand{\instructionsTwoROne}{send to recipient 2}
\newcommand{\instructionsOneRTwo}{read}

												% generate 3:2:1
												% generate card 1 for each set
\foreach \set in {\setNumberStart,...,\setNumberEnd}
{	\begin{center}
	\begin{figure}
		\resizebox{\paperwidth}{\paperheight}
		{%Generate QRK Number
\newcommand{\qrkNumber}{QRK$\;$M$\multipleNum_{\iterationNumber}\;\runDate:\begin{matrix*}[l]
\card \in \{ \cardNumberStart, \dots , \cardNumberEnd\} = S_{\set}\\
S_{\set} \subset \{ S_\setNumberStart, \dots , S_{\setNumberEnd}\}
\end{matrix*}$}
\begin{tikzpicture}[]
		% card
    	\draw[line width = 0.05cm] (-\halfWidth,-\halfHeight) rectangle (\halfWidth,\halfHeight);

    	% title
    	\node[anchor=north west] at (-\lrMarginStart,\udMarginStart) { \cardTitleSty{\cardTitle}};

    	% QRK Number
    	\node[anchor=south west] at (-\lrMarginStart, -\udMarginStart) { \tiny \qrkNumber};

    	% sender label
    	\ifthenelse{\senderOn = 1}
    					 % for the sender
    					 {\node[anchor=north east, rotate = 90, text = black!50!white] at (-\lrMarginStart,\senderNAnchor) {\personSty{\sender}};}
    					 % not for sender
    					 {\node[anchor=north east, rotate = 90, text = black!20!white] at (-\lrMarginStart,\senderNAnchor) {\personSty{\sender}};}

    	% recipient label
    	\ifthenelse{\recipientOn = 1}
    					 % for the recipient
    					 {\node[anchor=north east, rotate = -90, text = black!50!white ] at (\lrMarginStart, \recipientSAnchor) {\personSty{\recipient}};}
    					 % not for recipient
    					 {\node[anchor=north east, rotate = -90, text = black!20!white ] at (\lrMarginStart,\recipientSAnchor) {\personSty{\recipient}};}

    	% instructions - sender
    	\node[anchor = north west] at (\sendInstructionsX, \sendInstructionsY) { \instructionFormatting{\instructionsOneS} } ;
			\node[anchor = north west] at (\sendInstructionsX, \sendInstructionsY-0.4) { \instructionFormatting{\instructionsTwoS} } ;
						\node[anchor = north west] at (\sendInstructionsX, \sendInstructionsY-0.8) { \instructionFormatting{\instructionsThreeS} } ;
			% instructions - recipient
	\node[anchor = south east] at (\recipInstructionsX, \recipInstructionsY+0.4) { \instructionFormatting{\instructionsOneR} } ;
	\node[anchor = south east] at (\recipInstructionsX, \recipInstructionsY) { \instructionFormatting{\instructionsTwoR} } ;


\end{tikzpicture}
}
	\end{figure}
\end{center}
}

											}	%END Multiperson, !=  1:1/2:1/2:2/3:2
										}	% END Multiperson, !=  1:1/2:1/2:2
									}	% END Multiperson, not 1:1/2:1
								} 	% END Multiperson, not 1:1
							}	% END Multiperson Cards
						}	% END Card has 4 lines or is a multiperson card
					}	% END Card has 3 or more lines and is not sender
				}	% End 3 Line or More cards
			}	% END Card Has Multiple Lines 
					% - Not 2 line Sender
		} % End Else -> != one line cards		
	}% End For Loop through Each Card #
\end{document}
